\documentclass[12pt
,headinclude
,headsepline
,bibtotocnumbered
]{scrartcl}
\usepackage[paper=a4paper,left=25mm,right=25mm,top=25mm,bottom=25mm]{geometry} 
\usepackage[utf8]{inputenc}
\usepackage[ngerman]{babel}
\usepackage{graphicx}
\usepackage{multirow}
\usepackage{pdfpages}
%\usepackage{wrapfig}
\usepackage{placeins}
\usepackage{float}
\usepackage{flafter}
\usepackage{mathtools}
\usepackage{hyperref}
\usepackage{epstopdf}
\usepackage[miktex]{gnuplottex}
\usepackage[T1]{fontenc}
\usepackage{mhchem}
\usepackage{fancyhdr}
%\setlength{\mathindent}{0pt}
\usepackage{amssymb}
\usepackage[list=true, font=large, labelfont=bf, 
labelformat=brace, position=top]{subcaption}
\setlength{\parindent}{0mm}

\setlength{\parindent}{0mm}

\pagestyle{fancy}
\fancyhf{}
\lhead{Physikalische Geodäsie\\Übung 6: Schwereänderungen am BFO}
\rhead{Hsin-Feng Ho \\3378849}
\rfoot{Seite \thepage}
\begin{document}
	\begin{titlepage}
		\vspace{\fill}
		\title{\textbf{Physikalische Geodäsie \\ Übung 6: Schwereänderungen am BFO}}
		\vspace{5cm}
		\author{Hsin-Feng Ho\\
			3378849}
		\vspace{3cm}
		\maketitle
	\end{titlepage}
	\section{Signale in den Messdaten des supraleitenden Gravimeters}
	\section{Schwere, Gravitation und Zentrifugalbeschleunigung am BFO}
	\subsection{Gravitations- und Schwerepotential, Anziehung und Schwere}
	Gravitationspotential:
	\begin{align*}
		V=\frac{4}{3}\pi G\rho R^3\frac{1}{r}
	\end{align*}
	Zentrifugalpotential:
	\begin{align*}
		V_z=\frac{1}{2}\omega^2r^2\cos^2\phi
	\end{align*}
	Schwerepotential:
	\begin{align*}
		W=V+V_z
	\end{align*}
	\subsection{Störung der Erdrotation(Zentrifugalbeschleunigung)}
\end{document}

